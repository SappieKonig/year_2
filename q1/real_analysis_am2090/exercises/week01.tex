\documentclass{article}
\usepackage{amsmath,amsthm,amssymb}
\usepackage[thinc]{esdiff}

\newenvironment{myindentpar}[1]%
 {\begin{list}{}%
         {\setlength{\leftmargin}{#1}}%
         \item[]%
 }
 {\end{list}}


\begin{document}

\title{Real analysis exercises week 1} \maketitle


\textbf{Exercise 3.1}
\begin{myindentpar}{1cm}
Show that $d(x, y)=\lvert \frac{1}{x} - \frac{1}{y}\rvert$ defines a metric on $(0,\infty)$.

	Clearly, $\forall x, y \in \mathbb{R}: d(x, y)=0\iff x=y$ and $d(x, y)$, and

	$\forall x, y \in \mathbb{R}: d(x, y) \geq 0$, and

	$\forall x, y \in \mathbb{R}: d(x, y) = d(y, x)$.

	For the triangle inequality we use $\lvert a + b\rvert \leq \lvert a \rvert + \lvert b \rvert$.
	Taking $a = \frac{1}{x} - \frac{1}{z}$ and $b = \frac{1}{z} - \frac{1}{y}$ immediately yields the desired result.

\end{myindentpar}

\textbf{Exercise 3.3}
\begin{myindentpar}{1cm}
	Show that $\forall x, y \in \mathbb{R}: d(x, y)=0 \iff x=y$ and the triangle inequality are sufficient to denote a metric.

	Say there exist $x, y \in M$ such that $d(x, y)\neq d(y,x)$. This implies there are $x, y \in M$ such that $d(x,y) > d(y,x)$.
	Take $z=x$. Then $d(x,y) > d(x,x) + d(x,y)$, which is a contradiction. Hence $\forall x, y \in M$, $d(x,y)=d(y,x)$.

	Say there exist $x, z \in M$ such that $d(x,z) < 0$. Then taking $y=x$ and plugging it into the triangle inequality yields  $d(x,x) > d(x,z) + d(z,x)$,
	which is a contradiction. Hence $\forall x, y \in M$, we have $d(x,y) \geq 0$.

\end{myindentpar}

\textbf{Exercise 3.6}
\begin{myindentpar}{1cm}

	If $d$ is any metric on $M$, show that $\rho(x, y)=\sqrt{d(x,y)}$, $\sigma(x, y)=\frac{d(x, y)}{d(x, y) + 1}$, and $\tau(x, y)=min\{d(x,y), 1\}$ are also metrics on $M$.

	Given as all functions are $0$ iff $d(x,y)$ is $0$, this requirement is trivial. For the triangle equality it is then sufficient that the second derivative is negative.
	This is the case for the first two functions: $\diff*[2]{\rho}{d}=-\frac{1}{4}x^{-\frac{3}{2}}$, $\diff*[2]{\sigma}{d}=-\frac{2}{(1+x)^3}$ which are both negative.

	$min\{d(x,y), 1\}\leq d(x,y) \leq d(x,z) + d(z, y)$ which holds if both of $d(x,z)$ and $d(z,y)$ are less than $0$. If not, then either of them is one, which means
	$min\{d(x,y), 1\}\leq min\{d(x, z), 1\} + min\{d(z, y), 1\}$ definitely holds.

\end{myindentpar}

\textbf{Exercise 3.21}

Literally can't be bothered.

\end{document}
